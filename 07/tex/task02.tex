Although the multigrid method is most useful for 2D and 3D problems, 
let us learn how to use it with a simple 1-D problem.
NOTE: To keep things simple let us no longer use the sparse storage of 
the matrix (the compact storage without all the zeros), but simply use 
the full 2D matrix arrays. This will make it easier for you, and will 
give more insight, because you can then see the structure of the 
matrices more easily.
Start with a grid of $2^3+1=9$ points. The twice coarser grid will have 
$2^2+1=5$ grid points, and the next $2^1+1=3$ and finally the last will 
have $2^0+1=2$ points.

\paragraph{
    a) Construct the 2D matrices $R^{(3)}$, $R^{(2)}$, $R^{(1)}$ for 
    restriction to the next lower level (coarser grid).
} \ \\
    \\

\paragraph{
    b) Construct the 2D matrices $P^{(2)}$, $P^{(1)}$, $P^{(0)}$ for 
    prolongation to the next higher level (finer grid). Check that 
    the $P$ and $R$ matrices are (apart from a factor) each other’s 
    tranpose.
} \ \\
    \\

\paragraph{
    c) Take the $9\times9$ matrix of the previous section (or the one 
    from the 1D diffusion problem in the lecture notes), again in full 
    $9\times9$ shape, and use the $R^{(i)}$ and $P^{(i)}$ matrices to 
    create the restricted versions of that matrix. 
    Check that the result makes sense.
} \ \\
    \\

\paragraph{
    d) Construct a recursive subroutine/function that applies the 
    $V$-shaped multi-grid procedure using Jacobi iteration.
} \ \\
    \\

\paragraph{
    e) Compare to the solution you found in the previous exercise.
} \ \\
    \\
