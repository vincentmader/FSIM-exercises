In order to simulate the Riemann problem for the 1D Euler 
system, you'll have to implement a few modifications 
to your previous finite-volume solver: \\
\\
\begin{itemize}
    \item the set of conservative variables now is:
        \begin{equation}
            \vec q=
            \begin{bmatrix}
                \rho \\ \rho u \\ \rho E 
            \end{bmatrix}
        \end{equation}
        where $\rho E$ is the \textit{total energy density}:
        \begin{equation}
            \rho E:=e_\textnormal{int}+\frac{\rho u^2}{2}
        \end{equation}
        and $e_\textnormal{int}$ is the 
        \textit{internal energy density}.
    \item the physical fluxes are 
        \begin{equation}
            \vec q=
            \begin{bmatrix}
                \rho u \\ \rho u^2+P \\ u(\rho E+P) 
            \end{bmatrix}
        \end{equation}
    \item the equation of state is now adiabatic: You can 
        recover the value of the pressure from the total
        energy density:
        \begin{equation}
            P=(\gamma-1)\cdot\bigg(
                \rho E-\frac{\rho u^2}{2}
            \bigg)
        \end{equation}
        where $\gamma$ is the \textit{adiabatic index}.
        For this simulation you can set $\gamma=1.4$.
    \item In order to calculate the time step you can 
        use the formula given in the last exercise sheet.
        The only difference is that the sound speed $c_s$
        is not constant anymore:
        \begin{equation}
            c_s:=\frac{\gamma P}{\rho}
        \end{equation}
    \item the last two equations apply both at 
        cell centers and face centers.
    \item for setting \textit{Dirichlet} boundary 
        conditions, you can simply fill the first and 
        second ghost cells with the provided boundary
        values. In this case, there is no need to update 
        the ghost cells at every time step, just remember
        to fill them before entering the time loop.
\end{itemize}
You can now setup the following 1D Riemann problem: the 
$x$-grid goes from $x=0$ to $x=1$ and it is divided into 
$N_x=100$ cells. At $t=0$ the system is divided into 
\textit{left} and \textit{right} states:
\begin{itemize}
    \item left state ($x\leq0.5: \rho=1,p=1,u=0$)
    \item right state ($x>0.5: \rho=0.125,p=0.1,u=0$)
\end{itemize}
and $\gamma=1.4$. \\
\\
Dirichlet boundary conditions apply, and 
the boundary values are given by the initial L/R states. \\
\\
Note that 
\begin{equation}
    \rho E_\textnormal{L/R}(t=0)
    =\frac{p_\textnormal{L/R}(t=0)}{\gamma-1}
\end{equation}
\newpage

\paragraph{1. Solve this problem and plot the results for 
    $\rho(x)$, $u(x)$, and $p(x)$ at the final time
    ($t=0.2$).
} \ \\
    \\

\paragraph{2. Plot the time evolution of these quantities 
    in a $x$-$t$ diagram. (For instance, you can use 
    pyplot.imshow)
} \ \\
    \\

\paragraph{3. Redo the problem using $N_x=1000$. } \ \\
    \\

\paragraph{4. Explain the shape of the solution: where is 
    the contact discontinuity, where is the rarefaction 
    wave, and where is the shock wave?
} \ \\
    \\

\paragraph{5. Describe the differences that you observe 
    when increasing the resolution, especially with 
    respect to numerical diffusivity across the 
    rarefaction wave, the shock wave and the contact 
    discontinuity.
} \ \\
    \\

\paragraph{6. Experiment with the setup values of the 
    Riemann problem and try to find initial conditions 
    such that the outer (non-linear) sonic waves are both 
    shock waves travelling outwards. Plot the results for 
    $\rho(x)$, $u(x)$, and $p(x)$ at what you think it 
    may be a good final time.
} \ \\
    \\
