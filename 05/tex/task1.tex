\subsection{Unit redefinitions}
    Let us define
    \begin{equation}\label{redef1}
        \vec{r}'=\frac{\vec{r}}{\sigma} \qquad 
        E'=\frac{E}{\epsilon} \qquad 
        m'=\frac{m}{\mu}
    \end{equation}
    with $\mu=6.69\cdot 10^{-26}\mathrm{kg}$. For the velocity we can calculate
    \begin{equation}\label{redef_v}
        \sqrt{\frac{\epsilon}{\mu}}
        =:\nu \approx 157.0 \frac{\mathrm{m}}{\mathrm{s}} \qquad 
        \Rightarrow\qquad v'=\frac{\vec{v}}{\nu}
    \end{equation}
    and therefore 
    \begin{equation}\label{redef_t}
        \frac{\sigma}{\nu}
        =:\tau \approx2.2\cdot10^{-12}\mathrm{s}\qquad\Rightarrow
        \qquad t'=\frac{t}{\tau}
    \end{equation}
    later we'll need 
    \begin{equation}\label{redef_T}
        \frac{\epsilon}{k_B}=:\theta
        =120\mathrm{K}\qquad\Rightarrow\qquad T'=\frac{T}{\theta}
    \end{equation}

\subsection{Particle initialization}
    We implemented the Box-Muller-Method to get a Gaussian distribution. 
    For the velocities we multiplied the result from the Gaussian distribution 
    with the given sigma. This streches the velocity distribution, 
    s.t. the sigma of the velocity distribution is equal to the given sigma.
\subsubsection{Calculating a particle's acceleration during one timestep}
    We should update our neigborlist. In our version, we put all particle as 
    neigbours.
