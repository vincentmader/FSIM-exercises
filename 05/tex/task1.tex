\subsection{Unit redefinitions}
    Let us define
    \begin{equation}\label{redef1}
        \vec{r}'=\frac{\vec{r}}{\sigma} \qquad 
        E'=\frac{E}{\epsilon} \qquad 
        m'=\frac{m}{\mu}
    \end{equation}
    with $\mu=6.69\cdot 10^{-26}\mathrm{kg}$. 
    From this we can derive the following:
    \begin{align}
        \sqrt{\frac{\epsilon}{\mu}}
        =:\nu \approx 157.0 \frac{\mathrm{m}}{\mathrm{s}}
        \qquad&\Rightarrow\qquad v'=\frac{\vec{v}}{\nu}
        \label{redef_v} \\
        \frac{\sigma}{\nu}
        =:\tau \approx2.2\cdot10^{-12}\mathrm{s}
        \qquad&\Rightarrow\qquad t'=\frac{t}{\tau}
        \label{redef_t} \\
        \frac{\epsilon}{k_B}=:\theta=120\mathrm{K}
        \qquad&\Rightarrow\qquad T'=\frac{T}{\theta}
        \label{redef_T}
    \end{align}

\subsection{Particle initialization}
    First, we implement the Box-Muller-Method to get a Gaussian 
    distribution. \\
    \begin{lstlisting}
        double gaussian_rnd(void) {
            int r_1, r_2, n=32767;
            double r, phi, rand1, rand2, n_max=32767;

            rand1 = (rand() % n) / n_max;
            rand2 = (rand() % n) / n_max;

            r = sqrt(-2 * log(rand1));
            phi = 2 * 3.1415 * rand2;

            return r * cos(phi);
        }\end{lstlisting} 
    Then we initialize the velocities by drawing random numbers between 0 and 
    1 using the Gaussian distribution and multiply them with a given sigma,
    which stretches the velocity distribution. \\
    \begin{lstlisting}
        void initialize(particle *p, double L, int N1d, double sigma_v) {
            int n = 0;
            double dl = L / N1d; // L = 5*sigma N => d bar = dl = 5*sigma

            for(int i = 0; i < N1d; i++) {
              for(int j = 0; j < N1d; j++) {
                for(int k = 0; k < N1d; k++) {
                  // points on regular grid
                  p[n].pos[0] = i * dl;
                  p[n].pos[1] = j * dl;
                  p[n].pos[2] = k * dl;
                  // Gaussian-ditributed velocities
                  p[n].vel[0] = sigma_v * gaussian_rnd();
                  p[n].vel[1] = sigma_v * gaussian_rnd();
                  p[n].vel[2] = sigma_v * gaussian_rnd();

                  p[n].n_neighbors = N1d * N1d * N1d;
                  n++;
                }
              }
            }
        }\end{lstlisting}

\subsubsection{Calculating a particle's acceleration from the 
    forces acting on it
}
    Now, we define the function to calculate the potentials and forces for all 
    pairs of particles. \\
    \begin{lstlisting}
        void calc_forces(particle * p, int ntot, double boxsize, double rcut) {
          int n;
          double rcut2 = rcut * rcut;
          double r2, r, r6, r12, dr[3], acc[3], pot;
        
          // first, set all the accelerations and potentials to zero
          for (int i = 0; i < ntot; i++) {
            p[i].pot = 0;
            for (int k = 0; k < 3; k++) {
              p[i].acc[k] = 0;
            }
            // set neighbors
          }
        
          // sum over all distinct pairs
          for (int i = 0; i < ntot; i++) {

            for (int j = 0; j < ntot; j++) {
              if (i == j) {
                continue;
              }
        
              // Calculate squared distance
              double r2 = 0;
              for (int k = 0; k < 3; k++) {
                dr[k] = p[i].pos[k] - p[j].pos[k];
        
                // Ensure we find the shortest distance bewteen particles
                if (dr[k] > 0.5 * boxsize) {
                  dr[k] -= boxsize;
                }
        
                if (dr[k] < -0.5 * boxsize) {
                  dr[k] += boxsize;
                }
        
                r2 += dr[k] * dr[k];
              }
        
              // --- students ---
              if (r2 < 0.01) {
                continue;
              }
        
              if (r2 < rcut2) {
                r = sqrt(r2);
                r6 = r2 * r2 * r2;
                r12 = r6 * r6;
        
                // now calculate the Lennard-Jones potential for the pair
                pot = 4* (1 / r12 -1 /r6);
                p[i].pot += pot;
        
                // now calculate the Lennard-Jones force between the particles
                for (int k = 0; k < 3; k++) {
                  // m * a = F = -grad V = 4(pos[i]-pos[j]) *(12/ r^13 - 6 /r^7)
                  acc[k] = (p[i].pos[k] - p[j].pos[k]) * (48 / (r2*r12)-24/ (r2*r6));
                  p[i].acc[k] += acc[k];
                }
              }
              // --- end ---
            }
          }
        }\end{lstlisting}

