Consider the problem of a star of mass $M_*=M_\odot$ (where 
$M_\odot=\SI{1.99e30}{\kilogram}$ surrounded by a planet of 
$M_p=10^{-3} M_\odot$. At time $t=0$ the planet is located at 
coordinates $(1,0,0)$ in units of AU$=\SI{1.496e11}{\meter}$. The 
planet's velocity is $(0,0.5,0)$ in units of the Kepler velocity at the 
that location, which is 
$v_K(1\textnormal{ AU})=\SI{2.98e4}{\meter\per\second}$

\paragraph{Solve the Kepler orbit of this planet using numerical
    integration with the leapfrog algorithm. Find an appropriate 
    time step. Plot the result for the first few orbits.
} \ \\

\paragraph{Now integrate for 100 orbits and see how the orbit behaves.}

\paragraph{Repeat this with the RK2 and RK4 algorithms and see how the 
    system behaves. Discuss the difference to the leapfrog algorithm.
    For this, also plot the time evolution of the relative error 
    of the total energy and the time evolution of the total kinetic
    energy of the system.
}
