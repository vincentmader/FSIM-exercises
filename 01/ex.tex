\documentclass[11 pt]{article}

\usepackage[left=2cm,right=2cm,top=2cm,bottom=2cm]{geometry}
\usepackage{fancyhdr}
\usepackage{listings}
\usepackage{xcolor}


\definecolor{commentsColor}{rgb}{0.497495, 0.497587, 0.497464}
\definecolor{keywordsColor}{rgb}{0.000000, 0.000000, 0.635294}
\definecolor{stringColor}{rgb}{0.558215, 0.000000, 0.135316}
\lstset{ %
  backgroundcolor=\color{white},   % choose the background color; you must add \usepackage{color} or \usepackage{xcolor}
  basicstyle=\footnotesize,        % the size of the fonts that are used for the code
  breakatwhitespace=false,         % sets if automatic breaks should only happen at whitespace
  breaklines=true,                 % sets automatic line breaking
  captionpos=b,                    % sets the caption-position to bottom
  commentstyle=\color{commentsColor}\textit,    % comment style
  deletekeywords={...},            % if you want to delete keywords from the given language
  escapeinside={\%*}{*)},          % if you want to add LaTeX within your code
  extendedchars=true,              % lets you use non-ASCII characters; for 8-bits encodings only, does not work with UTF-8
  frame=tb,	                   	   % adds a frame around the code
  keepspaces=true,                 % keeps spaces in text, useful for keeping indentation of code (possibly needs columns=flexible)
  keywordstyle=\color{keywordsColor}\bfseries,       % keyword style
  language=Python,                 % the language of the code (can be overrided per snippet)
  otherkeywords={*,...},           % if you want to add more keywords to the set
  numbers=left,                    % where to put the line-numbers; possible values are (none, left, right)
  numbersep=5pt,                   % how far the line-numbers are from the code
  numberstyle=\tiny\color{commentsColor}, % the style that is used for the line-numbers
  rulecolor=\color{black},         % if not set, the frame-color may be changed on line-breaks within not-black text (e.g. comments (green here))
  showspaces=false,                % show spaces everywhere adding particular underscores; it overrides 'showstringspaces'
  showstringspaces=false,          % underline spaces within strings only
  showtabs=false,                  % show tabs within strings adding particular underscores
  stepnumber=1,                    % the step between two line-numbers. If it's 1, each line will be numbered
  stringstyle=\color{stringColor}, % string literal style
  tabsize=2,	                   % sets default tabsize to 2 spaces
  title=\lstname,                  % show the filename of files included with \lstinputlisting; also try caption instead of title
  columns=fixed                    % Using fixed column width (for e.g. nice alignment)
}

\pagestyle{fancy}
\fancyhf{}
\lhead{Fundamentals of Simulation Methods}
\chead{Exercise 01}
\rhead{V. Mader}
\setlength{\headheight}{15pt}


\begin{document}

    \section{Packing of numbers}
        Estimate how many numbers there are in the interval from 1.0 
        and 2.0; and in between the interval of 255.0 to 256.0, for 
        IEEE-754.

        \paragraph{a) single precision} \ \\
        A single precision (binary 32) floating point number has 1 sign bit,
        8 exponent bits and 24 bits of significand precision. Its range is 
        from $\pm1.18\cdot10^{-38}$ to $\pm3.4\cdot10^{38}$

        \paragraph{b) double precision} \ \\
        A double precision float can take on values from 
        $\pm2.23\cdot10^{-308}$ to $\pm1.80\cdot10^{308}$.
        

    \section{Pitfalls of integer \& floating point arithmetic}
        \paragraph{a)  Consider the following \textit{C/C++} code: } \ 
            \begin{lstlisting}
                int i = 7;
                float y = 2*(i/2);
                float z = 2*(i/2.);
                printf("%e %e \n", y,z) \end{lstlisting}
            The two variables $y$ and $z$ do not hold the same values, 
            since $i/2$ is evaluated to 3, while for $i/2.$ it is 2.5.
            Thus, multiplying with 2 once yields the integer 4, and once the 
            float 5.
        \paragraph{b) Again, consider the following \textit{C/C++} code: } \
            \begin{lstlisting}
                double a = 1.0e17;
                double b = -1.0e17;
                double c = 1.0;
                double x = (a + b) + c;
                double y = a + (b + c); \end{lstlisting}

    \newpage
    \section{Machine epsilon} 
        \paragraph{For the datatypes float, double \& long double, 
            determine the smallest number $\varepsilon_{min}$, such 
            that $1+\varepsilon_{min}$ still returns something different than 1.
        } \ \\ \\
        To find $\varepsilon_{min}$ for floats, we can utilize the following \textit{C} code 
        snippet: 
        \begin{lstlisting}
            #include <stdio.h>
            #include <stdbool.h>
            
            
            float find_epsilon () {
            
                float one = 1;
                float epsilon = 1;
                float new_epsilon;
            
                bool found_epsilon = false;
                while (!found_epsilon) {
                    new_epsilon = epsilon / 2.;
                    if (one + new_epsilon == one) {
                        return epsilon;
                    }
                    epsilon = new_epsilon;
                }
            }
            
            int main(void) {
                printf(
                    "%e \n", find_epsilon()
                );
            }
        \end{lstlisting}
        All occurences of \textit{float} can be switched out for \textit{double} 
        or \textit{long double} to get the other values of  $\varepsilon_{min}$. \\
        \\
        This yields:
        \begin{table}[h!]
            \begin{center}
                \begin{tabular}{r|l}
                    type & $\varepsilon_{min}$ \\
                    \hline
                    long double & 0 \\
                    double & $\approx10^{-16}$ \\
                    float & $\approx10^{-7}$ 
                \end{tabular}
            \end{center}
        \end{table} 
        \paragraph{
            Evaluate and print out $1+\varepsilon$. Do you see something strange?
        } \ \\
    

    \newpage
    \section{Near-cancellation of numbers}
        Consider the following function:
        \begin{equation}
            f(x)=\frac{x+e^{-x}-1}{x^2}
        \end{equation}

        \paragraph{a)}
            Determine $\lim_{x\to0}f(x)$

        \paragraph{b)}
            Write a computer program that asks for a value of $x$ from the 
            user and then prints $f(x)$.

        \paragraph{c)}
            For small $x>0$ this evaluation goes wrong. Determine 
            experimentally at which values of $x$ the formula goes wrong.

        \paragraph{d)}
            Explain why this happens.

        \paragraph{e)}
            Add an if-clause to the program such that for small values the 
            function is evaluated in another way that does not break down, so 
            that for all positive values of $x$ the program produces a 
            reasonable result.

\end{document}

