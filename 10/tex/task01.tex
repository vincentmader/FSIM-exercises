In this exercise you are going to write and
use an Eulerian finite-volume code (based on
the Godunov method) to solve the 1D equations
of isothermal hydrodynamics:
\begin{align}
    0&=
    \pder{\rho}{t}+
    \pder{\rho u}{x} \\ \label{Euler1}
    0&=
    \pder{(\rho u)}{t}+
    \pder{[\rho u^2+P]}{x} \\ \label{Euler2}
\end{align}
where $\rho$ is the density, $u$ the velocity,
and $P$ the pressure. \\
\\
The system is written in
conservative form. The set of conservative
variables is
\begin{equation}
    \vec q=
    \begin{bmatrix}
        \rho \\ \rho u
    \end{bmatrix}
\end{equation}
and the fluxes are
\begin{equation}
    \vec F_\textnormal{Euler}=
    \begin{bmatrix}
        \rho u \\ \rho u^2+P
    \end{bmatrix}
\end{equation}
Consider a small linear perturbation on
the density:
\begin{equation}
    \rho(x,t)=
    \rho_0+\delta\rho(x,t)
\end{equation}
We can Fourier-decompose $\delta\rho(x,t)$
into modes of the type
\begin{equation}
    \delta\rho(x,t)
    =A\cdot e^{i(kx-\omega t)}
\end{equation}

\paragraph{1. Derive the dispersion relation
    between $\omega$ and $k$.
} \ \\
    \\
We assume, that $\rho_0$ is in equilibrium and therefore the spacial and the time derivatives are zero. In equilibrium $u=0$. If we perturb $\rho$, we need also perturb $u$ and $P$.

The perturbed quantities satisfying the Euler equations and we get from \eqref{Euler1}
 \begin{align}
 	0 &= \partial_t(\rho_0 + \delta \rho) + \partial_x [(\rho_0 + \delta \rho) \delta u]\\
 	&\approx \partial_t (\delta \rho)+ \rho_0 \partial_x \delta u
 	\label{App1}
 \end{align}
In the last step we used, that $\delta \rho \delta u = \mathcal{O}(\delta^2)$ and therefore negligible.
Instead of \eqref{Euler2} we use the equivalent
\begin{equation}
	0 = \partial_t u + (u \partial_x)u + \frac{1}{\rho} \partial_x P \end{equation}

and get with perturbations to
\begin{align}
	0 &= \partial_t \delta u + (\delta u \partial_x)\delta u + \frac{1}{\rho_0 + \delta \rho} \partial_x (P+\delta P) \\
	& \approx \partial_t \delta u + \frac{1}{\rho}\partial_x P
	\label{App2}
\end{align}

by taking the time derivative of \eqref{App1} and the spacial deriavative and subtracting \eqref{App1} from \eqref{App2} we get:
\begin{align}
	0&=\partial_t(\partial_t (\delta \rho)+ \rho_0 partial_x \delta u) - \partial_x (
	\partial_t \delta u + \frac{1}{\rho}\partial_x P)\\
	&= \partial_t^2\delta \rho - \partial_x^2 \delta P\\
    &= \partial_t^2\delta \rho - c_s^2\partial_x^2 \delta \rho
\end{align}
where we used $ P = c_s^2 / \rho$.
With $\rho = A e^{i(kx-\omega t)}$ we get

\begin{equation}
	0=\omega^2-k^2 \qquad\Rightarrow\qquad \omega= \pm k
\end{equation}



\paragraph{2. Explain the meaning of the
    solution
} \ \\
    \\
The dispersion relation $\omega(k)$ is linear in $k$. This means, the perturbation is stable for all times.

\paragraph{3. Argue, on the basis of these
    solutions, that the isothermal sound
    speed is $c_s=\sqrt{P/\rho}$, so that
    $P=\rho c_s^2$.
} \ \\
    \\

\paragraph{4. Show that the system of
    isothermal Euler equations has the
    Eigen-values $\lambda_{1}=u-c_s$,
    $\lambda_{2}=u+c_s$ associated with the
    Eigen-vectors $\vec K_{1}=[1, u-c_s]^T$
    and $\vec K_{2}=[1, u+c_s]^T$
}
